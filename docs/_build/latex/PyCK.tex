% Generated by Sphinx.
\def\sphinxdocclass{report}
\documentclass[letterpaper,10pt,english]{sphinxmanual}
\usepackage[utf8]{inputenc}
\DeclareUnicodeCharacter{00A0}{\nobreakspace}
\usepackage[T1]{fontenc}
\usepackage{babel}
\usepackage{times}
\usepackage[Bjarne]{fncychap}
\usepackage{longtable}
\usepackage{sphinx}
\usepackage{multirow}


\title{PyCK Documentation}
\date{March 01, 2012}
\release{0.4.0}
\author{Kashif Iftikhar}
\newcommand{\sphinxlogo}{}
\renewcommand{\releasename}{Release}
\makeindex

\makeatletter
\def\PYG@reset{\let\PYG@it=\relax \let\PYG@bf=\relax%
    \let\PYG@ul=\relax \let\PYG@tc=\relax%
    \let\PYG@bc=\relax \let\PYG@ff=\relax}
\def\PYG@tok#1{\csname PYG@tok@#1\endcsname}
\def\PYG@toks#1+{\ifx\relax#1\empty\else%
    \PYG@tok{#1}\expandafter\PYG@toks\fi}
\def\PYG@do#1{\PYG@bc{\PYG@tc{\PYG@ul{%
    \PYG@it{\PYG@bf{\PYG@ff{#1}}}}}}}
\def\PYG#1#2{\PYG@reset\PYG@toks#1+\relax+\PYG@do{#2}}

\def\PYG@tok@gd{\def\PYG@tc##1{\textcolor[rgb]{0.63,0.00,0.00}{##1}}}
\def\PYG@tok@gu{\let\PYG@bf=\textbf\def\PYG@tc##1{\textcolor[rgb]{0.50,0.00,0.50}{##1}}}
\def\PYG@tok@gt{\def\PYG@tc##1{\textcolor[rgb]{0.00,0.25,0.82}{##1}}}
\def\PYG@tok@gs{\let\PYG@bf=\textbf}
\def\PYG@tok@gr{\def\PYG@tc##1{\textcolor[rgb]{1.00,0.00,0.00}{##1}}}
\def\PYG@tok@cm{\let\PYG@it=\textit\def\PYG@tc##1{\textcolor[rgb]{0.25,0.50,0.56}{##1}}}
\def\PYG@tok@vg{\def\PYG@tc##1{\textcolor[rgb]{0.73,0.38,0.84}{##1}}}
\def\PYG@tok@m{\def\PYG@tc##1{\textcolor[rgb]{0.13,0.50,0.31}{##1}}}
\def\PYG@tok@mh{\def\PYG@tc##1{\textcolor[rgb]{0.13,0.50,0.31}{##1}}}
\def\PYG@tok@cs{\def\PYG@tc##1{\textcolor[rgb]{0.25,0.50,0.56}{##1}}\def\PYG@bc##1{\colorbox[rgb]{1.00,0.94,0.94}{##1}}}
\def\PYG@tok@ge{\let\PYG@it=\textit}
\def\PYG@tok@vc{\def\PYG@tc##1{\textcolor[rgb]{0.73,0.38,0.84}{##1}}}
\def\PYG@tok@il{\def\PYG@tc##1{\textcolor[rgb]{0.13,0.50,0.31}{##1}}}
\def\PYG@tok@go{\def\PYG@tc##1{\textcolor[rgb]{0.19,0.19,0.19}{##1}}}
\def\PYG@tok@cp{\def\PYG@tc##1{\textcolor[rgb]{0.00,0.44,0.13}{##1}}}
\def\PYG@tok@gi{\def\PYG@tc##1{\textcolor[rgb]{0.00,0.63,0.00}{##1}}}
\def\PYG@tok@gh{\let\PYG@bf=\textbf\def\PYG@tc##1{\textcolor[rgb]{0.00,0.00,0.50}{##1}}}
\def\PYG@tok@ni{\let\PYG@bf=\textbf\def\PYG@tc##1{\textcolor[rgb]{0.84,0.33,0.22}{##1}}}
\def\PYG@tok@nl{\let\PYG@bf=\textbf\def\PYG@tc##1{\textcolor[rgb]{0.00,0.13,0.44}{##1}}}
\def\PYG@tok@nn{\let\PYG@bf=\textbf\def\PYG@tc##1{\textcolor[rgb]{0.05,0.52,0.71}{##1}}}
\def\PYG@tok@no{\def\PYG@tc##1{\textcolor[rgb]{0.38,0.68,0.84}{##1}}}
\def\PYG@tok@na{\def\PYG@tc##1{\textcolor[rgb]{0.25,0.44,0.63}{##1}}}
\def\PYG@tok@nb{\def\PYG@tc##1{\textcolor[rgb]{0.00,0.44,0.13}{##1}}}
\def\PYG@tok@nc{\let\PYG@bf=\textbf\def\PYG@tc##1{\textcolor[rgb]{0.05,0.52,0.71}{##1}}}
\def\PYG@tok@nd{\let\PYG@bf=\textbf\def\PYG@tc##1{\textcolor[rgb]{0.33,0.33,0.33}{##1}}}
\def\PYG@tok@ne{\def\PYG@tc##1{\textcolor[rgb]{0.00,0.44,0.13}{##1}}}
\def\PYG@tok@nf{\def\PYG@tc##1{\textcolor[rgb]{0.02,0.16,0.49}{##1}}}
\def\PYG@tok@si{\let\PYG@it=\textit\def\PYG@tc##1{\textcolor[rgb]{0.44,0.63,0.82}{##1}}}
\def\PYG@tok@s2{\def\PYG@tc##1{\textcolor[rgb]{0.25,0.44,0.63}{##1}}}
\def\PYG@tok@vi{\def\PYG@tc##1{\textcolor[rgb]{0.73,0.38,0.84}{##1}}}
\def\PYG@tok@nt{\let\PYG@bf=\textbf\def\PYG@tc##1{\textcolor[rgb]{0.02,0.16,0.45}{##1}}}
\def\PYG@tok@nv{\def\PYG@tc##1{\textcolor[rgb]{0.73,0.38,0.84}{##1}}}
\def\PYG@tok@s1{\def\PYG@tc##1{\textcolor[rgb]{0.25,0.44,0.63}{##1}}}
\def\PYG@tok@gp{\let\PYG@bf=\textbf\def\PYG@tc##1{\textcolor[rgb]{0.78,0.36,0.04}{##1}}}
\def\PYG@tok@sh{\def\PYG@tc##1{\textcolor[rgb]{0.25,0.44,0.63}{##1}}}
\def\PYG@tok@ow{\let\PYG@bf=\textbf\def\PYG@tc##1{\textcolor[rgb]{0.00,0.44,0.13}{##1}}}
\def\PYG@tok@sx{\def\PYG@tc##1{\textcolor[rgb]{0.78,0.36,0.04}{##1}}}
\def\PYG@tok@bp{\def\PYG@tc##1{\textcolor[rgb]{0.00,0.44,0.13}{##1}}}
\def\PYG@tok@c1{\let\PYG@it=\textit\def\PYG@tc##1{\textcolor[rgb]{0.25,0.50,0.56}{##1}}}
\def\PYG@tok@kc{\let\PYG@bf=\textbf\def\PYG@tc##1{\textcolor[rgb]{0.00,0.44,0.13}{##1}}}
\def\PYG@tok@c{\let\PYG@it=\textit\def\PYG@tc##1{\textcolor[rgb]{0.25,0.50,0.56}{##1}}}
\def\PYG@tok@mf{\def\PYG@tc##1{\textcolor[rgb]{0.13,0.50,0.31}{##1}}}
\def\PYG@tok@err{\def\PYG@bc##1{\fcolorbox[rgb]{1.00,0.00,0.00}{1,1,1}{##1}}}
\def\PYG@tok@kd{\let\PYG@bf=\textbf\def\PYG@tc##1{\textcolor[rgb]{0.00,0.44,0.13}{##1}}}
\def\PYG@tok@ss{\def\PYG@tc##1{\textcolor[rgb]{0.32,0.47,0.09}{##1}}}
\def\PYG@tok@sr{\def\PYG@tc##1{\textcolor[rgb]{0.14,0.33,0.53}{##1}}}
\def\PYG@tok@mo{\def\PYG@tc##1{\textcolor[rgb]{0.13,0.50,0.31}{##1}}}
\def\PYG@tok@mi{\def\PYG@tc##1{\textcolor[rgb]{0.13,0.50,0.31}{##1}}}
\def\PYG@tok@kn{\let\PYG@bf=\textbf\def\PYG@tc##1{\textcolor[rgb]{0.00,0.44,0.13}{##1}}}
\def\PYG@tok@o{\def\PYG@tc##1{\textcolor[rgb]{0.40,0.40,0.40}{##1}}}
\def\PYG@tok@kr{\let\PYG@bf=\textbf\def\PYG@tc##1{\textcolor[rgb]{0.00,0.44,0.13}{##1}}}
\def\PYG@tok@s{\def\PYG@tc##1{\textcolor[rgb]{0.25,0.44,0.63}{##1}}}
\def\PYG@tok@kp{\def\PYG@tc##1{\textcolor[rgb]{0.00,0.44,0.13}{##1}}}
\def\PYG@tok@w{\def\PYG@tc##1{\textcolor[rgb]{0.73,0.73,0.73}{##1}}}
\def\PYG@tok@kt{\def\PYG@tc##1{\textcolor[rgb]{0.56,0.13,0.00}{##1}}}
\def\PYG@tok@sc{\def\PYG@tc##1{\textcolor[rgb]{0.25,0.44,0.63}{##1}}}
\def\PYG@tok@sb{\def\PYG@tc##1{\textcolor[rgb]{0.25,0.44,0.63}{##1}}}
\def\PYG@tok@k{\let\PYG@bf=\textbf\def\PYG@tc##1{\textcolor[rgb]{0.00,0.44,0.13}{##1}}}
\def\PYG@tok@se{\let\PYG@bf=\textbf\def\PYG@tc##1{\textcolor[rgb]{0.25,0.44,0.63}{##1}}}
\def\PYG@tok@sd{\let\PYG@it=\textit\def\PYG@tc##1{\textcolor[rgb]{0.25,0.44,0.63}{##1}}}

\def\PYGZbs{\char`\\}
\def\PYGZus{\char`\_}
\def\PYGZob{\char`\{}
\def\PYGZcb{\char`\}}
\def\PYGZca{\char`\^}
\def\PYGZsh{\char`\#}
\def\PYGZpc{\char`\%}
\def\PYGZdl{\char`\$}
\def\PYGZti{\char`\~}
% for compatibility with earlier versions
\def\PYGZat{@}
\def\PYGZlb{[}
\def\PYGZrb{]}
\makeatother

\begin{document}

\maketitle
\tableofcontents
\phantomsection\label{index::doc}


Contents:


\chapter{Introduction}
\label{README:introduction}\label{README::doc}\label{README:welcome-to-pyck-s-documentation}\label{README:id1}
\textbf{PyCK} (Python Code Karigar); prounounced `\emph{pick}` is/would be a web development framework
aiming to provide an easy to use yet powerful and flexible web framework for python developers.

Documentation for the project can be found at \href{http://packages.python.org/PyCK/}{PyCK Documentation}


\section{Background}
\label{README:background}
Of course, there are already a lot of great frameworks present for python like \href{http://docs.pylonsproject.org/en/latest/docs/pylons.html}{Pylons}, \href{https://www.djangoproject.com/}{Django}, \href{http://docs.pylonsproject.org/en/latest/docs/pyramid.html}{Pyramid}, \href{http://bluebream.zope.org/}{BlueBream (Previously Zope)}, \href{http://turbogears.org/}{TurboGears2} that provide really cool
features for developing web applications in python, I feel like there was still something missing for my taste. So you can say that this project is starting to satisfy a personal itch.


\subsection{Why Create PyCK?}
\label{README:why-create-pyck}
Of all the frameworks mentioned above I liked some features of one framework and other features of another framework. Unfortunately I wasn't able to find all those features I like in any existing framework. For example Django probably is the most popular framework in python and for good reason, it has pluggable apps, extensible extensions like its admin panel, a lot of useful documentation and great community support. One of its biggest strengths is that a lot of apps are available for it which can be ``plugged'' into a new project to get things started really quickly. The problem is that django is very ``opnionated'', the choices like ORM, templating language, URL dispatching mechanism are all made for the developer and you are mostly stuck with them unless you are willing to put in a lot of extra effort into it.

Consider the following scenario, my favorite ORM is \href{http://www.sqlalchemy.org/}{SQLAlchemy} and for good reason. I can develop command line applications, traditional GUIs in GUI toolkits like Qt, GTK etc all using SQLAlchemy as the ORM to interact with the database. Now if I decide to use django, I need to learn and use its ORM, why can't I use the one I already am familiar and proficient with? Though I can but that breaks a lot of stuff in Django.

Pylons and Pyramid on the other hand are very ``non-opinionated'' frameworks. They both are very flexible and I really like the way things are done in these frameworks. I can use SQLAlchemy or any other ORM like SQLObject etc if I like. I can choose the templating language I want to use (which BTW would be \href{http://www.makotemplates.org/}{Mako}). I can choose the URL handling mechanism (in Pyramid) be it URLDispatch or Traversal. But this flexiblity comes at a cost, building ``ready-made'' components for such frameworks isn't easily possible because we are not sure what the framework user will pick as technologies. So having pluggable apps or pre-built admin panels etc because tough.

So the solution? at least for me; I decided to build a framework based on Pyramid that makes the choices for the developers. If your choices are the same as mine, this framework would be ideal for you. Or if you are a new developer looking into python frameworks you can start here (just not right now since the work has only started yet).


\section{Discussion on various aspects of PyCK}
\label{README:discussion-on-various-aspects-of-pyck}
{\hyperref[form-validation-library-choice:forms-library-choice]{\emph{The Problem (The Why? of Form Library)}}}

{\hyperref[pyck-project-structure:pyck-project-structure]{\emph{Structure of a PyCK Project}}}

{\hyperref[pluggable-apps-howto:pluggable-apps]{\emph{Pluggable application in PyCK}}}


\section{Feature Plan}
\label{README:feature-plan}
And what exactly are the choices?
\begin{itemize}
\item {} 
Use \textbf{SQLAlchemy as the ORM}

\item {} 
Use \textbf{Mako as the templating language}

\item {} 
Use \textbf{URLDispatch as the resource location} - URL to code mapping mechanism

\item {} 
Design should support \textbf{Pluggable applications} similar to Djano

\item {} 
Should have easily \textbf{extendible components} like an admin panel, etc

\item {} 
Allow web applications to be easily \textbf{Themable}

\item {} 
Use \href{http://dojotoolkit.org/}{Dojo} for UI components, AJAX etc

\item {} 
Ability to easily specify \textbf{separate view templates for mobile devices} (using Dojox.mobile)

\item {} 
\textbf{Automatic form generation from database/SQLAlchemy models} (looking into possible options like sprox, formalchemy, wtforms, deform, etc)

\end{itemize}


\chapter{Changes}
\label{changes:changes}\label{changes::doc}\label{changes:id1}
This document lists the changes as versions progress


\section{Whats new in 0.4}
\label{changes:whats-new-in-0-4}\begin{itemize}
\item {} 
CRUDController - Enables automatic CRUD interface generation from database models. The {\hyperref[pyck-controllers:pyck.controllers.CRUDController]{\code{pyck.controllers.CRUDController}}} allows you to quickly enable CRUD interface for any database model you like. To use CRUD controller at minimum these steps must be followed.
\begin{quote}
\begin{enumerate}
\item {} 
Create a sub-class of the CRUDController and set model (for which you want to have CRUD) and database session:

\begin{Verbatim}[commandchars=\\\{\}]
\PYG{k+kn}{from} \PYG{n+nn}{pyck.controllers} \PYG{k+kn}{import} \PYG{n}{CRUDController}
\PYG{k+kn}{from} \PYG{n+nn}{myapp.models} \PYG{k+kn}{import} \PYG{n}{MyModel}\PYG{p}{,} \PYG{n}{DBSession}

\PYG{k}{class} \PYG{n+nc}{MyCRUDController}\PYG{p}{(}\PYG{n}{CRUDController}\PYG{p}{)}\PYG{p}{:}
    \PYG{n}{model} \PYG{o}{=} \PYG{n}{MyModel}
    \PYG{n}{db\PYGZus{}session} \PYG{o}{=} \PYG{n}{DBSession}\PYG{p}{(}\PYG{p}{)}
\end{Verbatim}

\item {} 
In your application's routes settings, specify the url where the CRUD interface should be displayed. You can use the {\hyperref[pyck-controllers:pyck.controllers.add_crud_handler]{\code{pyck.controllers.add\_crud\_handler()}}} method for it. For example in your \_\_init\_\_.py (if you're enabling CRUD for a model without your main project) or in your routes.py (if you're enabling CRUD for a model within an app in your project) put code like:

\begin{Verbatim}[commandchars=\\\{\}]
\PYG{k+kn}{from} \PYG{n+nn}{pyck.controllers} \PYG{k+kn}{import} \PYG{n}{add\PYGZus{}crud\PYGZus{}handler}
\PYG{k+kn}{from} \PYG{n+nn}{controllers.views} \PYG{k+kn}{import} \PYG{n}{MyCRDUController}

\PYG{c}{\PYGZsh{} Place this with the config.add\PYGZus{}route calls}
\PYG{n}{add\PYGZus{}crud\PYGZus{}handler}\PYG{p}{(}\PYG{n}{config}\PYG{p}{,} \PYG{l+s}{'}\PYG{l+s}{mymodel\PYGZus{}crud}\PYG{l+s}{'}\PYG{p}{,} \PYG{l+s}{'}\PYG{l+s}{/mymodel}\PYG{l+s}{'}\PYG{p}{,} \PYG{n}{WikiCRUDController}\PYG{p}{)}
\end{Verbatim}

\end{enumerate}

and that's all you need to do to get a fully operation CRUD interface. Take a look at the newapp sample app in demos for a working CRUD example in the Wiki app.
\end{quote}

\end{itemize}


\section{Whats new in 0.3}
\label{changes:whats-new-in-0-3}\begin{itemize}
\item {} 
Model Forms - Ability to generate forms automatically from database models. We now have a {\hyperref[pyck-forms:pyck.forms.model_form]{\code{pyck.forms.model\_form()}}} function that behaves exactly like \code{wtforms.ext.sqlalchemy.orm.model\_form()} but uses {\hyperref[pyck-forms:pyck.forms.Form]{\code{pyck.forms.Form}}} as its base class. The benefit is that you get all the features present in pyck forms in your model form (like, as\_p and as\_table rendering of your form and CSRF protection). Using a model form is quite easy, for example:

\begin{Verbatim}[commandchars=\\\{\}]
\PYG{k+kn}{from} \PYG{n+nn}{pyck.forms} \PYG{k+kn}{import} \PYG{n}{model\PYGZus{}form}
\PYG{k+kn}{from} \PYG{n+nn}{myapp.models} \PYG{k+kn}{import} \PYG{n}{User}
\PYG{n}{UserForm} \PYG{o}{=} \PYG{n}{model\PYGZus{}form}\PYG{p}{(}\PYG{n}{User}\PYG{p}{)}
\end{Verbatim}

Of course, you can then sub-class this UserForm class to add further validators or modifications if you like. Later in a view (considering you've not subclassed UserForm) you can use this form as:

\begin{Verbatim}[commandchars=\\\{\}]
\PYG{n}{f} \PYG{o}{=} \PYG{n}{UserForm}\PYG{p}{(}\PYG{n}{request}\PYG{o}{.}\PYG{n}{POST}\PYG{p}{,} \PYG{n}{request\PYGZus{}obj}\PYG{o}{=}\PYG{n}{request}\PYG{p}{,} \PYG{n}{use\PYGZus{}csrf\PYGZus{}protection}\PYG{o}{=}\PYG{n+nb+bp}{True}\PYG{p}{)}
\end{Verbatim}

and it will work exactly like a normal pyck Form.

\item {} 
A more operational blog app in the newapp given in demos that uses the model\_form feature to add blog posts.

\end{itemize}


\section{Whats new in 0.2.4}
\label{changes:whats-new-in-0-2-4}\begin{itemize}
\item {} 
Automated CSRF Protection in forms. While disabled by default (to maintain compatibility with WTForms), CSRF protection can be enabled for a form by passing the form two extra keyword arguments \textbf{request\_obj} and \textbf{use\_csrf\_protection} set to \textbf{True} when initializing it. For example:

\begin{Verbatim}[commandchars=\\\{\}]
\PYG{n}{f} \PYG{o}{=} \PYG{n}{ContactForm}\PYG{p}{(}\PYG{n}{request}\PYG{o}{.}\PYG{n}{POST}\PYG{p}{,} \PYG{n}{request\PYGZus{}obj}\PYG{o}{=}\PYG{n}{request}\PYG{p}{,} \PYG{n}{use\PYGZus{}csrf\PYGZus{}protection}\PYG{o}{=}\PYG{n+nb+bp}{True}\PYG{p}{)}
\end{Verbatim}

\item {} 
Form objects now have an as\_table {\hyperref[pyck-forms:pyck.forms.Form.as_table]{\code{pyck.forms.Form.as\_table()}}} method that allows displaying the form in a table similar to the {\hyperref[pyck-forms:pyck.forms.Form.as_p]{\code{pyck.forms.Form.as\_p()}}} method added in previous release. This method also accepts labels and errors positions (left, right, top, bottom) and optionally allows you to insert the html \textless{}table\textgreater{} tag within the method instead of putting it in your template by setting \textbf{include\_table\_tag parameter} to \textbf{True}

\end{itemize}


\section{Whats new in 0.2.3}
\label{changes:whats-new-in-0-2-3}
Till now almost all updates were to the scaffold generated by a PyCK project, so in a sense till now PyCK could be considered another scraffold for Pyramid. With this version, things are starting to change a bit.
\begin{itemize}
\item {} 
A new package {\hyperref[pyck-forms:module-pyck.forms]{\code{pyck.forms}}} that serves as a wrapper on top of WTForms (will try to maintain code usage compatibility with wtforms) so instead of using normal \textbf{wtforms.Form} class instances, PyCK developers can use {\hyperref[pyck-forms:pyck.forms.Form]{\code{pyck.forms.Form}}} instances in the same way. But these forms come with some additional features
\begin{itemize}
\item {} 
Currently the form can be display using html p tags using {\hyperref[pyck-forms:pyck.forms.Form.as_p]{\code{pyck.forms.Form.as\_p()}}} method. This method supports displaying labels and validation errors on either direction of the field control (top, bottom, left, right).

\item {} 
The associated sample app code has been updated along with new app scaffold to use pyck.forms, the code already has become much simpler.

\item {} 
It is important to note that these forms can be used in the same way as WTForms so if you want to layout your form the way you want (as you normally do in WTForms); you are still able to do it.

\end{itemize}

\item {} 
Basic tests have been implemented for {\hyperref[pyck-forms:module-pyck.forms]{\code{pyck.forms}}} and nosetests are being used for automated testing. Keeping the code quality high is one of the aims here so I'll try to write tests for all of the additions to pyck itself.

\end{itemize}


\section{Whats new in 0.2.2}
\label{changes:whats-new-in-0-2-2}\begin{itemize}
\item {} 
Sessions support - Sessions come pre-configured now with a new PyCK project and the sample included has also been updated accordingly

\item {} 
Forms support - Initial support for forms using WTForms has landed. Keeping with the structure forms are defined within a forms package inside the application package.

\item {} 
A newly created project (and the sample project) now contains a contact form demonstrating forms usage.
\begin{itemize}
\item {} 
Additionally forms also have CSRF (Cross Site Request Forgery) protection

\end{itemize}

\item {} 
Flash messaging support is also in. Look at the contact form example (specifically its template and the home and base templates) to see flash messages in action.

\end{itemize}

\textbf{Whats next?} Focus now is to make forms more easy to use within PyCK. Upcoming versions are expected to contain more enhancements related to forms.


\section{Whats new in 0.2.1}
\label{changes:whats-new-in-0-2-1}
Some code refactoring to ease up a few things
\begin{itemize}
\item {} 
Moved sys.path addition settings to a seperate function named load\_project\_settings in project's \_\_init\_\_.py. This function is called by \_\_init\_\_.py's main function to load project specific settings and also called by the populate script. So the code is at one place instead of two places.

\item {} 
For apps, moved the RenameTables SQA MetaBase to the model package's \_\_init\_\_.py so its a bit hidden from the developer as the developer just sees:

\begin{Verbatim}[commandchars=\\\{\}]
\PYG{k+kn}{from} \PYG{n+nn}{.} \PYG{k+kn}{import} \PYG{n}{DBSession}\PYG{p}{,} \PYG{n}{Base}
\end{Verbatim}

in the model definition files. This also makes importing these into multiple model files much easier (since again the code is at a single location now)

\item {} 
In the \_\_init\_\_.py of every model package (apps or the main project alike), we now import the models defined by that project/app and include them in the \_\_all\_\_ list so that instead of importing like:

\begin{Verbatim}[commandchars=\\\{\}]
\PYG{k+kn}{from} \PYG{n+nn}{myapp.models.models} \PYG{k+kn}{import} \PYG{n}{MyModel}
\end{Verbatim}

now we can use:

\begin{Verbatim}[commandchars=\\\{\}]
\PYG{k+kn}{from} \PYG{n+nn}{myapp.models} \PYG{k+kn}{import} \PYG{n}{MyModel}
\end{Verbatim}

\end{itemize}


\section{Whats new in 0.2.0}
\label{changes:whats-new-in-0-2-0}\begin{itemize}
\item {} 
tables created from models in apps are automatically prefixed by app name. For example: if you have an app named blog and it has a model Post where you have specified:

\begin{Verbatim}[commandchars=\\\{\}]
\PYG{n}{\PYGZus{}\PYGZus{}tablename\PYGZus{}\PYGZus{}} \PYG{o}{=} \PYG{l+s}{'}\PYG{l+s}{posts}\PYG{l+s}{'}
\end{Verbatim}

it will automatically be created as \textbf{blog\_posts} in the database. Your access to the table through the model remained same without any changes.

\item {} 
Once you run python setup.py develop for your new project, a new command for creating an app becomes availabe to you. Instead of copying the sample app provided and adjusting it, now the whole struture is created for you. For details see

{\hyperref[adding-apps:adding-apps]{\emph{Adding an Application}}}

This feature is the reason that the version number bumped upto 0.2 :-)

\end{itemize}


\section{Whats new in 0.1.6}
\label{changes:whats-new-in-0-1-6}\begin{itemize}
\item {} 
First fully operational version with pluggable apps along with their database models etc.

\end{itemize}


\chapter{Installation}
\label{installation:installation}\label{installation::doc}\label{installation:id1}
Installing \textbf{PyCK} is as easy as installing any other python package using either \textbf{easy\_install} or \textbf{pip}. It is a good idea to setup pyck inside a virtual environment. Here is how:
\begin{enumerate}
\item {} 
(optional) Install pip if you don't already have it:

\begin{Verbatim}[commandchars=\\\{\}]
easy\_install pip
\end{Verbatim}

\item {} 
Install virtualenv if you don't have it already:

\begin{Verbatim}[commandchars=\\\{\}]
pip install virtualenv
\end{Verbatim}

\item {} 
Create a virtual environment for your pip applications:

\begin{Verbatim}[commandchars=\\\{\}]
virtualenv --no-site-packages pyckenv
\end{Verbatim}

\item {} 
Activate your new virtual environment:

\begin{Verbatim}[commandchars=\\\{\}]
source pyckenv/bin/activate
\end{Verbatim}

Anytime you want to deactivate the environment just issue the command \textbf{deactivate}

\item {} 
Install pyck:

\begin{Verbatim}[commandchars=\\\{\}]
pip install pyck
\end{Verbatim}

This is the only step you actually need to do, if you already have easy\_install or pip, installing pyck is as easy as \textbf{easy\_install pyck} or \textbf{pip install pyck}, however using a virtual environment sandboxes the rest of the system nicely for  you.

\end{enumerate}


\chapter{Starting a Project}
\label{start-project:starting-a-project}\label{start-project::doc}\label{start-project:start-project}
Once you have successfully installed PyCK and have optionally activated a virtaul environment, you are ready to start your first project. To start your first project, follow these steps.
\begin{enumerate}
\item {} 
Move to the folder where you want to create your project's base folder.

\item {} 
Create the project structure using:

\begin{Verbatim}[commandchars=\\\{\}]
pcreate -t pyck myproject
\end{Verbatim}

Replace myproject with the name of your project. This creates the basic structure for your project including configuration files for both development and production, adding project package and sub foldres for controllers/view, models, templates, scripts, tests etc. Your project has support for pluggable applications which can be placed in the apps folder under your project's package. The generated code already contains one app there with its basic structure ready to use. If you like, you can copy this structure to create more apps.

\item {} 
Move into your newly created project's base folder:

\begin{Verbatim}[commandchars=\\\{\}]
cd myproject
\end{Verbatim}

\item {} 
Run the setup script with develop parameter to install any dependant packages if they are missing:

\begin{Verbatim}[commandchars=\\\{\}]
python setup.py develop
\end{Verbatim}

\item {} 
Now you should create any database models under \textless{}yourprojectname\textgreater{}/models/models.py that you wish to use, once done (or even if you don't want to create any models yet), run:

\begin{Verbatim}[commandchars=\\\{\}]
populate\_myproject development.ini
\end{Verbatim}

Remember to replace myproject with the name of your project. This script automatically creates tables in the database specified in development.ini configuration file. By default a SQLite file with the same name as your project is used, you can change the \textbf{sqlalchemy.url} parameter in development.ini (or production.ini in case you're changing the backend DB for production). Remember that this populate script needs to be executed again in case you add/change your models in either your main project or in any of its sub-applications. When re-running the populate script, you may need to delete the existing records/tables from your DB for this script to execute without errors.

\item {} 
Now your application is ready to start. From now on you only need this step to start serving your application through the built-in web server provided by PyCK/Pyramid. To start serving your application run:

\begin{Verbatim}[commandchars=\\\{\}]
pserve development.ini --reload
\end{Verbatim}

You may see output similar to:

\begin{Verbatim}[commandchars=\\\{\}]
Starting subprocess with file monitor
Starting server in PID 8191.
Starting HTTP server on http://0.0.0.0:6543
\end{Verbatim}

Note the listening port and IP can be changed in development.ini.

\item {} 
Open your browser and type:

\begin{Verbatim}[commandchars=\\\{\}]
http://localhost:6543
\end{Verbatim}

and congratulations you should see the initial page saing Welcome and displaying a link to \textbf{My Application} in the middle. Clicking this link opens up a view from My Application just for demonstrating the use of pluggable apps.

\end{enumerate}

From here on you can start developing your project however you like :-)


\chapter{Adding an Application}
\label{adding-apps:adding-an-application}\label{adding-apps:adding-apps}\label{adding-apps::doc}
Once you run python setup.py develop for your new project, a new command for creating an app becomes availabe to you. Instead of copying the sample app provided and adjusting it, now the whole struture is created for you. Lets assume you have named your project \textbf{myproject}. After running:

\begin{Verbatim}[commandchars=\\\{\}]
python setup.py develop
\end{Verbatim}

You will be able to use a new command named \textbf{myproject\_newapp}. This command takes just one argument - the name of the new app and automatically creates all its structure under the apps folder. You still need to enable it by adding it to the \textbf{enabled\_apps} list in \textbf{apps/\_\_init\_\_.py}.

\begin{Verbatim}[commandchars=\\\{\}]
myproject\_newapp blog
\end{Verbatim}

The structure created for you is ready to use as soon as you add the apps name in the enabled\_apps list but you might want to create your models and add some other stuff first.


\chapter{Pluggable application in PyCK}
\label{pluggable-apps-howto:pluggable-application-in-pyck}\label{pluggable-apps-howto:pluggable-apps}\label{pluggable-apps-howto::doc}
Pluggable apps are just like normal pyck (or pyramid) project with a few modifications. Scaffolds for generating a pluggable application structure will be coming shortly. For the time being, this page describes how a pluggable application needs to behave to be able to successfully integrated into a PyCK project with minimal effort.


\section{Implement application\_routes function in \_\_init\_\_.py}
\label{pluggable-apps-howto:implement-application-routes-function-in-init-py}
Taking a blog app as an example, in your app's \_\_init\_\_.py, implement a function like:

\begin{Verbatim}[commandchars=\\\{\}]
\PYG{k}{def} \PYG{n+nf}{application\PYGZus{}routes}\PYG{p}{(}\PYG{n}{config}\PYG{p}{)}\PYG{p}{:}
    \PYG{n}{config}\PYG{o}{.}\PYG{n}{add\PYGZus{}route}\PYG{p}{(}\PYG{l+s}{'}\PYG{l+s}{blog.home}\PYG{l+s}{'}\PYG{p}{,} \PYG{l+s}{'}\PYG{l+s}{/}\PYG{l+s}{'}\PYG{p}{)}
    \PYG{n}{config}\PYG{o}{.}\PYG{n}{add\PYGZus{}route}\PYG{p}{(}\PYG{l+s}{'}\PYG{l+s}{blog.about}\PYG{l+s}{'}\PYG{p}{,} \PYG{l+s}{'}\PYG{l+s}{/about}\PYG{l+s}{'}\PYG{p}{)}

    \PYG{n}{config}\PYG{o}{.}\PYG{n}{add\PYGZus{}static\PYGZus{}view}\PYG{p}{(}\PYG{l+s}{'}\PYG{l+s}{static}\PYG{l+s}{'}\PYG{p}{,} \PYG{l+s}{'}\PYG{l+s}{static}\PYG{l+s}{'}\PYG{p}{,} \PYG{n}{cache\PYGZus{}max\PYGZus{}age}\PYG{o}{=}\PYG{l+m+mi}{3600}\PYG{p}{)}
\end{Verbatim}

And in your main project's \_\_init\_\_.py you can add the routes from this application using:

\begin{Verbatim}[commandchars=\\\{\}]
\PYG{n}{config}\PYG{o}{.}\PYG{n}{include}\PYG{p}{(}\PYG{n}{application\PYGZus{}routes}\PYG{p}{,} \PYG{n}{route\PYGZus{}prefix}\PYG{o}{=}\PYG{l+s}{'}\PYG{l+s}{/blog}\PYG{l+s}{'}\PYG{p}{)}
\end{Verbatim}

This takes care of accessing your app correctly from within the main project.


\section{Implement a populate\_app function to your app's scripts/populate.py script}
\label{pluggable-apps-howto:implement-a-populate-app-function-to-your-app-s-scripts-populate-py-script}
This function will be called by the main project's populate script to automatically add tables and
records for the app to the project's database:

\begin{Verbatim}[commandchars=\\\{\}]
\PYG{k}{def} \PYG{n+nf}{populate\PYGZus{}app}\PYG{p}{(}\PYG{n}{engine}\PYG{p}{,} \PYG{n}{db\PYGZus{}session}\PYG{p}{)}\PYG{p}{:}
    \PYG{n}{Base}\PYG{o}{.}\PYG{n}{metadata}\PYG{o}{.}\PYG{n}{create\PYGZus{}all}\PYG{p}{(}\PYG{n}{engine}\PYG{p}{)}
    \PYG{k}{with} \PYG{n}{transaction}\PYG{o}{.}\PYG{n}{manager}\PYG{p}{:}
        \PYG{n}{model} \PYG{o}{=} \PYG{n}{Post}\PYG{p}{(}\PYG{l+s}{'}\PYG{l+s}{Test}\PYG{l+s}{'}\PYG{p}{,} \PYG{l+s}{'}\PYG{l+s}{Just testing}\PYG{l+s}{'}\PYG{p}{)}
        \PYG{n}{db\PYGZus{}session}\PYG{o}{.}\PYG{n}{add}\PYG{p}{(}\PYG{n}{model}\PYG{p}{)}
\end{Verbatim}


\chapter{Structure of a PyCK Project}
\label{pyck-project-structure::doc}\label{pyck-project-structure:pyck-project-structure}\label{pyck-project-structure:structure-of-a-pyck-project}
Here is the structure of a typical PyCK project (as though of till now) assuming a project named \textbf{combined\_apps} and one pluggable application named \textbf{blog}

\begin{Verbatim}[commandchars=\\\{\}]
combined\_apps/
\textbar{}-- CHANGES.txt
\textbar{}-- MANIFEST.in
\textbar{}-- README.txt
\textbar{}-- combined\_apps                         (The main project folder containing all the code)
\textbar{}   \textbar{}-- \_\_init\_\_.py                   (Project's init file containing initialization code and routes)
\textbar{}
\textbar{}   \textbar{}-- apps                          (This folder contains any pluggable apps)
\textbar{}   \textbar{}   \textbar{}-- \_\_init\_\_.py               (This file contains enabled apps list and some utility stuff)
\textbar{}   \textbar{}   {}`-- blog                      (A sample blog app that is pluggable)
\textbar{}   \textbar{}       \textbar{}-- \_\_init\_\_.py          (Apps initialization code \& the application\_routes function)
\textbar{}   \textbar{}       \textbar{}-- controllers          (The controllers folder containing all the controllers for the app)
\textbar{}   \textbar{}       \textbar{}   \textbar{}-- \_\_init\_\_.py
\textbar{}   \textbar{}       \textbar{}-- models               (Application's models)
\textbar{}   \textbar{}       \textbar{}   \textbar{}-- \_\_init\_\_.py
\textbar{}   \textbar{}       \textbar{}-- scripts              (Other scripts including the populate script containing the populate\_app function)
\textbar{}   \textbar{}       \textbar{}   \textbar{}-- \_\_init\_\_.py
\textbar{}   \textbar{}       \textbar{}   \textbar{}-- populate.py
\textbar{}   \textbar{}       \textbar{}-- static               (Application specific static media like images, css, javascript, etc)
\textbar{}   \textbar{}       \textbar{}-- templates            (Application's template - normally (but not compulsarily) in mako templating language)
\textbar{}   \textbar{}       {}`-- tests                (Application's Tests)
\textbar{}   \textbar{}           \textbar{}-- \_\_init\_\_.py
\textbar{}   \textbar{}-- controllers                  (Main project's controllers)
\textbar{}   \textbar{}   \textbar{}-- \_\_init\_\_.py
\textbar{}   \textbar{}-- models                       (Main project's models)
\textbar{}   \textbar{}   \textbar{}-- \_\_init\_\_.py
\textbar{}   \textbar{}-- scripts                      (Main project's scripts including the populate script)
\textbar{}   \textbar{}   \textbar{}-- \_\_init\_\_.py
\textbar{}   \textbar{}   \textbar{}-- populate.py
\textbar{}   \textbar{}-- static                       (Main proejct's static media)
\textbar{}   \textbar{}-- templates                    (Main project's templates)
\textbar{}   {}`-- tests                        (Main project's tests)
\textbar{}       \textbar{}-- \_\_init\_\_.py
\textbar{}-- combined\_apps.db                 (Project's DB if using sqlite)
\textbar{}-- combined\_apps.egg-info           (Project's egg/setup related files)
\textbar{}-- development.ini                  (Project's configuration for development setup)
\textbar{}-- production.ini                   (Project's configuration for production deployment)
\textbar{}-- setup.cfg                        (Configuration for the setup script)
{}`-- setup.py                              (Project's setup script)
\end{Verbatim}


\chapter{The Problem (The Why? of Form Library)}
\label{form-validation-library-choice:the-problem-the-why-of-form-library}\label{form-validation-library-choice:forms-library-choice}\label{form-validation-library-choice::doc}
So why use a form validation and generation library and not just write plain forms customized to one's heart's content in the view templates? Here are some reasons:
\begin{itemize}
\item {} 
Not all forms need extensive customization, backend admin panels normally need CRUD interfaces

\item {} 
Front-end forms visible to site visitors normally do need customization so even if we use a form library, it should provide easy customization options in case we need it

\item {} 
Building lots of forms and doing their validation tends to get boring pretty quickly, so if it can be done effectively and quickly, it is appreciated

\item {} 
Centralized validation (most of it derived from the model) eliminates code duplication for validation at schema/model level and at form level

\item {} 
And there may be more reasons but right now I can't think of any

\end{itemize}


\section{The choices}
\label{form-validation-library-choice:the-choices}
In the project introduction, I mentioned something along the lines:
\begin{itemize}
\item {} 
\textbf{Automatic form generation from database/SQLAlchemy models} (looking into possible options like sprox, formalchemy, wtforms, deform, etc)

\end{itemize}

Did some digging and finalized two contenders, \textbf{sprox} and \textbf{wtforms}


\subsection{WTForms}
\label{form-validation-library-choice:wtforms}
WTForms is small easy to use and easy to learn. Has good structure (IMHO) for validation and object population etc. More complex structures can be built on top of that. But it is a lower level library, perhaps similar on the level to the much complex (and much difficult to develop and sometimes use) toscawidgets library.


\subsubsection{Problems}
\label{form-validation-library-choice:problems}\begin{itemize}
\item {} 
Just basic forms library, no higher auto-generated model-backed CRUD intefaces (which sprox offers).

\end{itemize}


\subsection{Sprox}
\label{form-validation-library-choice:sprox}
Provides whole CRUD interfaces with just few lines. Uses ToscaWidgets (Not TW2) behind the scenes. Has some Dojo support which we're also aiming at. Basically Sprox is something that fits in nicely for our requirements. But ....... :-(


\subsubsection{Problems}
\label{form-validation-library-choice:id1}\begin{itemize}
\item {} 
Sprox is based on ToscaWidgets and the scene with ToscaWidgets is not quite clear. The ToscaWidgets 2 branch has existed for years now but still Toscawidgets 1 is used more commonly. Toscawidgets (both 1 \& 2) while aiming to be very powerful and flexible also suffers from the complexity that comes with such aims.

\item {} 
Developing custom widgets for toscawidgets isn't all that easy; I have written some dojo charting widgets for TW2. While TW and TW2 tries to handle a lot of things like resource injection etc, this complexity is a deterrant.

\end{itemize}


\section{Conclusion}
\label{form-validation-library-choice:conclusion}
Using sprox reduces a lot of time that would be otherwise spent on building another implementation similar to Sprox but which uses WTForms as its backend instead of ToscaWidgets and isn't just coupled with SQLAlchemy as the ORM (at this time).

But I am inclined to choose \textbf{WTForms} though this means ending up building another sprox-like implementation. I believe that building upon WTForms won't be that painful and since WTForms does provide better methods to populate objects and read values from object, its forms are less coupled with the backend ORM engine. Though for PyCK we will be using SQLAlchemy, this does allow some reusability that can be contributed back to the WTForms project. Also I seem to have a gut feeling about using WTForms ;-)

\textbf{So the decision is to use WTForms and build a Sprox-like library on top of it that provides simple and Dojo based forms and other display widgets (tables, etc).}


\chapter{The pyck.forms Package}
\label{pyck-forms:module-pyck.forms}\label{pyck-forms:pyck-forms}\label{pyck-forms::doc}\label{pyck-forms:the-pyck-forms-package}\index{pyck.forms (module)}\phantomsection\label{pyck-forms:module-pyck.forms}\index{pyck.forms (module)}
The PyCK Forms Package
\index{Form (class in pyck.forms)}

\begin{fulllineitems}
\phantomsection\label{pyck-forms:pyck.forms.Form}\pysiglinewithargsret{\strong{class }\code{pyck.forms.}\bfcode{Form}}{\emph{formdata=None}, \emph{obj=None}, \emph{prefix='`}, \emph{request\_obj=None}, \emph{use\_csrf\_protection=False}, \emph{**kwargs}}{}
The Form base class, extends from WTForms.Form

This class provides some additional features to the base wtforms.Form class. These include:
\begin{itemize}
\item {} \begin{description}
\item[{Methods to render the form in HTML}] \leavevmode\begin{itemize}
\item {} 
as\_p to render form using p tags

\item {} 
as\_table to render using html table

\end{itemize}

\end{description}

\item {} 
CSRF protection availability

\item {} 
TODO: Since WTForms wants presentation attributes (like rows and cols for textarea) to be set on

\end{itemize}

instantiated form instances (not the class itself), allow some way to specify field attributes
that are then read at field display time. Possibly a field\_attrs dict??? May be reimplementing sub-classes of some fields to make use of it would be a could idea??
\index{as\_p() (pyck.forms.Form method)}

\begin{fulllineitems}
\phantomsection\label{pyck-forms:pyck.forms.Form.as_p}\pysiglinewithargsret{\bfcode{as\_p}}{\emph{labels='top'}, \emph{errors='right'}}{}
Output each form field as html \textbf{p} tags. By default labels are displayed on top of the form fields
and validation erros are displayed on the right of the form fields. Both these behaviors can be
changed by settings values for the labels and errors parameters.

Values can be left, top, right or bottom
\begin{quote}\begin{description}
\item[{Parameters}] \leavevmode\begin{itemize}
\item {} 
\textbf{labels} -- Placement of labels relative to the field

\item {} 
\textbf{errors} -- Placement of validation errors (if any) relative to the field

\end{itemize}

\end{description}\end{quote}

\end{fulllineitems}

\index{as\_table() (pyck.forms.Form method)}

\begin{fulllineitems}
\phantomsection\label{pyck-forms:pyck.forms.Form.as_table}\pysiglinewithargsret{\bfcode{as\_table}}{\emph{labels='left'}, \emph{errors='top'}, \emph{include\_table\_tag=False}}{}
Output the form as HTML Table, optionally add the table tags too if include\_table\_tag is set to True (default False)
\begin{quote}\begin{description}
\item[{Parameters}] \leavevmode\begin{itemize}
\item {} 
\textbf{labels} -- Placement of labels relative to the field

\item {} 
\textbf{errors} -- Placement of validation errors (if any) relative to the field

\item {} 
\textbf{include\_table\_tag} -- Whether to include the html \textless{}table\textgreater{} and \textless{}/table\textgreater{} tags in the output or not

\end{itemize}

\end{description}\end{quote}

\end{fulllineitems}

\index{validate() (pyck.forms.Form method)}

\begin{fulllineitems}
\phantomsection\label{pyck-forms:pyck.forms.Form.validate}\pysiglinewithargsret{\bfcode{validate}}{}{}
Validate form fields and check for CSRF token match if use\_csrf\_protection was set to true when
initializing the form.

\end{fulllineitems}


\end{fulllineitems}

\index{model\_form() (in module pyck.forms)}

\begin{fulllineitems}
\phantomsection\label{pyck-forms:pyck.forms.model_form}\pysiglinewithargsret{\code{pyck.forms.}\bfcode{model\_form}}{\emph{model}, \emph{base\_class=\textless{}class `pyck.forms.form.Form'\textgreater{}}, \emph{only=None}, \emph{exclude=None}, \emph{field\_args=None}, \emph{converter=None}}{}
A Wrapper around \code{wtforms.ext.sqlalchemy.orm.model\_form()} function to facilitate creating model
forms using a wtforms compatible model\_form call but using {\hyperref[pyck-forms:pyck.forms.Form]{\code{pyck.forms.Form}}}
Create a wtforms Form for a given SQLAlchemy model class:

\begin{Verbatim}[commandchars=\\\{\}]
\PYG{k+kn}{from} \PYG{n+nn}{pyck.forms} \PYG{k+kn}{import} \PYG{n}{model\PYGZus{}form}
\PYG{k+kn}{from} \PYG{n+nn}{myapp.models} \PYG{k+kn}{import} \PYG{n}{User}
\PYG{n}{UserForm} \PYG{o}{=} \PYG{n}{model\PYGZus{}form}\PYG{p}{(}\PYG{n}{User}\PYG{p}{)}
\end{Verbatim}
\begin{quote}\begin{description}
\item[{Parameters}] \leavevmode\begin{itemize}
\item {} 
\textbf{model} -- A SQLAlchemy mapped model class.

\item {} 
\textbf{base\_class} -- Base form class to extend from. Must be a \code{wtforms.Form} subclass.

\item {} 
\textbf{only} -- An optional iterable with the property names that should be included in
the form. Only these properties will have fields.

\item {} 
\textbf{exclude} -- An optional iterable with the property names that should be excluded
from the form. All other properties will have fields.

\item {} 
\textbf{field\_args} -- An optional dictionary of field names mapping to keyword arguments used
to construct each field object.

\item {} 
\textbf{converter} -- A converter to generate the fields based on the model properties. If
not set, \code{ModelConverter} is used.

\end{itemize}

\end{description}\end{quote}

\end{fulllineitems}


Here is an example usage to demonstrate the improvements:

\begin{Verbatim}[commandchars=\\\{\}]
\PYG{k+kn}{from} \PYG{n+nn}{pyck.forms} \PYG{k+kn}{import} \PYG{n}{Form}
\PYG{k+kn}{from} \PYG{n+nn}{wtforms} \PYG{k+kn}{import} \PYG{n}{TextField}\PYG{p}{,} \PYG{n}{validators}
\PYG{k}{class} \PYG{n+nc}{MyForm}\PYG{p}{(}\PYG{n}{Form}\PYG{p}{)}\PYG{p}{:}
    \PYG{n}{name} \PYG{o}{=} \PYG{n}{TextField}\PYG{p}{(}\PYG{l+s}{"}\PYG{l+s}{Name}\PYG{l+s}{"}\PYG{p}{,} \PYG{p}{[}\PYG{n}{validators}\PYG{o}{.}\PYG{n}{required}\PYG{p}{(}\PYG{p}{)}\PYG{p}{]}\PYG{p}{)}
    \PYG{n}{fname} \PYG{o}{=} \PYG{n}{TextField}\PYG{p}{(}\PYG{l+s}{"}\PYG{l+s}{Father}\PYG{l+s}{'}\PYG{l+s}{s Name}\PYG{l+s}{"}\PYG{p}{,} \PYG{p}{[}\PYG{n}{validators}\PYG{o}{.}\PYG{n}{required}\PYG{p}{(}\PYG{p}{)}\PYG{p}{]}\PYG{p}{)}

\PYG{n}{myform} \PYG{o}{=} \PYG{n}{MyForm}\PYG{p}{(}\PYG{p}{)}
\PYG{n+nb}{str}\PYG{p}{(}\PYG{n}{myform}\PYG{o}{.}\PYG{n}{as\PYGZus{}p}\PYG{p}{(}\PYG{p}{)}\PYG{p}{)}
\PYG{l+s}{r'}\PYG{l+s}{\PYGZbs{}}\PYG{l+s}{n\textless{}p\textgreater{}}\PYG{l+s}{\PYGZbs{}}\PYG{l+s}{n\textless{}label for=}\PYG{l+s}{"}\PYG{l+s}{name}\PYG{l+s}{"}\PYG{l+s}{\textgreater{}Name\textless{}/label\textgreater{}\textless{}br /\textgreater{} \textless{}input id=}\PYG{l+s}{"}\PYG{l+s}{name}\PYG{l+s}{"}\PYG{l+s}{ name=}\PYG{l+s}{"}\PYG{l+s}{name}\PYG{l+s}{"}\PYG{l+s}{ type=}\PYG{l+s}{"}\PYG{l+s}{text}\PYG{l+s}{"}\PYG{l+s}{ value=}\PYG{l+s}{"}\PYG{l+s}{"}\PYG{l+s}{ /\textgreater{} \textless{}/p\textgreater{}}\PYG{l+s}{\PYGZbs{}}\PYG{l+s}{n}\PYG{l+s}{\PYGZbs{}}\PYG{l+s}{n\textless{}p\textgreater{}}\PYG{l+s}{\PYGZbs{}}\PYG{l+s}{n\textless{}label for=}\PYG{l+s}{"}\PYG{l+s}{fname}\PYG{l+s}{"}\PYG{l+s}{\textgreater{}Father}\PYG{l+s+se}{\PYGZbs{}'}\PYG{l+s}{s Name\textless{}/label\textgreater{}\textless{}br /\textgreater{} \textless{}input id=}\PYG{l+s}{"}\PYG{l+s}{fname}\PYG{l+s}{"}\PYG{l+s}{ name=}\PYG{l+s}{"}\PYG{l+s}{fname}\PYG{l+s}{"}\PYG{l+s}{ type=}\PYG{l+s}{"}\PYG{l+s}{text}\PYG{l+s}{"}\PYG{l+s}{ value=}\PYG{l+s}{"}\PYG{l+s}{"}\PYG{l+s}{ /\textgreater{} \textless{}/p\textgreater{}}\PYG{l+s}{\PYGZbs{}}\PYG{l+s}{n}\PYG{l+s}{'}
\PYG{n}{myform}\PYG{o}{.}\PYG{n}{validate}\PYG{p}{(}\PYG{p}{)}
\PYG{n+nb+bp}{False}
\PYG{n+nb}{str}\PYG{p}{(}\PYG{n}{myform}\PYG{o}{.}\PYG{n}{as\PYGZus{}p}\PYG{p}{(}\PYG{n}{labels}\PYG{o}{=}\PYG{l+s}{'}\PYG{l+s}{left}\PYG{l+s}{'}\PYG{p}{,} \PYG{n}{errors}\PYG{o}{=}\PYG{l+s}{'}\PYG{l+s}{right}\PYG{l+s}{'}\PYG{p}{)}\PYG{p}{)}
\PYG{l+s}{'}\PYG{l+s+se}{\PYGZbs{}n}\PYG{l+s}{\textless{}p\textgreater{}}\PYG{l+s+se}{\PYGZbs{}n}\PYG{l+s}{\textless{}label for=}\PYG{l+s}{"}\PYG{l+s}{name}\PYG{l+s}{"}\PYG{l+s}{\textgreater{}Name\textless{}/label\textgreater{} \textless{}input id=}\PYG{l+s}{"}\PYG{l+s}{name}\PYG{l+s}{"}\PYG{l+s}{ name=}\PYG{l+s}{"}\PYG{l+s}{name}\PYG{l+s}{"}\PYG{l+s}{ type=}\PYG{l+s}{"}\PYG{l+s}{text}\PYG{l+s}{"}\PYG{l+s}{ value=}\PYG{l+s}{"}\PYG{l+s}{"}\PYG{l+s}{ /\textgreater{} \textless{}span class=}\PYG{l+s}{"}\PYG{l+s}{errors}\PYG{l+s}{"}\PYG{l+s}{\textgreater{}This field is required.\textless{}/span\textgreater{} \textless{}/p\textgreater{}}\PYG{l+s+se}{\PYGZbs{}n}\PYG{l+s+se}{\PYGZbs{}n}\PYG{l+s}{\textless{}p\textgreater{}}\PYG{l+s+se}{\PYGZbs{}n}\PYG{l+s}{\textless{}label for=}\PYG{l+s}{"}\PYG{l+s}{fname}\PYG{l+s}{"}\PYG{l+s}{\textgreater{}Father}\PYG{l+s+se}{\PYGZbs{}'}\PYG{l+s}{s Name\textless{}/label\textgreater{} \textless{}input id=}\PYG{l+s}{"}\PYG{l+s}{fname}\PYG{l+s}{"}\PYG{l+s}{ name=}\PYG{l+s}{"}\PYG{l+s}{fname}\PYG{l+s}{"}\PYG{l+s}{ type=}\PYG{l+s}{"}\PYG{l+s}{text}\PYG{l+s}{"}\PYG{l+s}{ value=}\PYG{l+s}{"}\PYG{l+s}{"}\PYG{l+s}{ /\textgreater{} \textless{}span class=}\PYG{l+s}{"}\PYG{l+s}{errors}\PYG{l+s}{"}\PYG{l+s}{\textgreater{}This field is required.\textless{}/span\textgreater{} \textless{}/p\textgreater{}}\PYG{l+s+se}{\PYGZbs{}n}\PYG{l+s}{'}
\end{Verbatim}

Within a template, all you need to do is to just call the form's rendering function, for example:

\begin{Verbatim}[commandchars=\\\{\}]
\$\PYGZob{}myform.as\_p() \textbar{} n\PYGZcb{}
\end{Verbatim}


\chapter{The pyck.controllers Package}
\label{pyck-controllers:module-pyck.controllers}\label{pyck-controllers:the-pyck-controllers-package}\label{pyck-controllers::doc}\label{pyck-controllers:pyck-controllers}\index{pyck.controllers (module)}\index{CRUDController (class in pyck.controllers)}

\begin{fulllineitems}
\phantomsection\label{pyck-controllers:pyck.controllers.CRUDController}\pysiglinewithargsret{\strong{class }\code{pyck.controllers.}\bfcode{CRUDController}}{\emph{request}}{}
Enables automatic CRUD interface generation from database models. The {\hyperref[pyck-controllers:pyck.controllers.CRUDController]{\code{pyck.controllers.CRUDController}}} allows you to quickly enable CRUD interface for any database model you like. To use CRUD controller at minimum these steps must be followed.
\begin{enumerate}
\item {} 
Create a sub-class of the CRUDController and set model (for which you want to have CRUD) and database session:

\begin{Verbatim}[commandchars=\\\{\}]
\PYG{k+kn}{from} \PYG{n+nn}{pyck.controllers} \PYG{k+kn}{import} \PYG{n}{CRUDController}
\PYG{k+kn}{from} \PYG{n+nn}{myapp.models} \PYG{k+kn}{import} \PYG{n}{MyModel}\PYG{p}{,} \PYG{n}{DBSession}

\PYG{k}{class} \PYG{n+nc}{MyCRUDController}\PYG{p}{(}\PYG{n}{CRUDController}\PYG{p}{)}\PYG{p}{:}
    \PYG{n}{model} \PYG{o}{=} \PYG{n}{MyModel}
    \PYG{n}{db\PYGZus{}session} \PYG{o}{=} \PYG{n}{DBSession}\PYG{p}{(}\PYG{p}{)}
\end{Verbatim}

\item {} 
In your application's routes settings, specify the url where the CRUD interface should be displayed. You can use the {\hyperref[pyck-controllers:pyck.controllers.add_crud_handler]{\code{pyck.controllers.add\_crud\_handler()}}} method for it. For example in your \_\_init\_\_.py (if you're enabling CRUD for a model without your main project) or in your routes.py (if you're enabling CRUD for a model within an app in your project) put code like:

\begin{Verbatim}[commandchars=\\\{\}]
\PYG{k+kn}{from} \PYG{n+nn}{pyck.controllers} \PYG{k+kn}{import} \PYG{n}{add\PYGZus{}crud\PYGZus{}handler}
\PYG{k+kn}{from} \PYG{n+nn}{controllers.views} \PYG{k+kn}{import} \PYG{n}{MyCRDUController}

\PYG{c}{\PYGZsh{} Place this with the config.add\PYGZus{}route calls}
\PYG{n}{add\PYGZus{}crud\PYGZus{}handler}\PYG{p}{(}\PYG{n}{config}\PYG{p}{,} \PYG{l+s}{'}\PYG{l+s}{mymodel\PYGZus{}crud}\PYG{l+s}{'}\PYG{p}{,} \PYG{l+s}{'}\PYG{l+s}{/mymodel}\PYG{l+s}{'}\PYG{p}{,} \PYG{n}{WikiCRUDController}\PYG{p}{)}
\end{Verbatim}

\end{enumerate}

and that's all you need to do to get a fully operation CRUD interface. Take a look at the newapp sample app in demos for a working CRUD example in the Wiki app.

\textbf{Configuration Options}

These parameters are to be set as class properties in a sub-class of CRUDController
\begin{quote}\begin{description}
\item[{Parameters}] \leavevmode\begin{itemize}
\item {} 
\textbf{model} -- (Required) The SQLAlchemy model class for which the CRUD interface is desired

\item {} 
\textbf{db\_session} -- (Required) The SQLAlchemy database session that should be used for operations

\item {} 
\textbf{friendly\_name} -- A human-friendly name of the model. If given this is used in the templates instead of the model name

\item {} 
\textbf{add\_edit\_exclude} -- A list of fields that should not be displayed in add or edit operations

\item {} 
\textbf{list\_recs\_per\_page} -- Number of records to display in a listing page. Default 10.

\item {} 
\textbf{list\_only} -- List of fields that are to be displayed on listing page, all other fields are ignored.

\item {} 
\textbf{list\_exclude} -- List of fields to be exluded in listing page

\item {} 
\textbf{list\_actions} -- 
list of actions that are to be displayed in listing page, example:

\begin{Verbatim}[commandchars=\\\{\}]
\PYG{n}{list\PYGZus{}actions} \PYG{o}{=} \PYG{p}{[}
        \PYG{p}{\PYGZob{}}\PYG{l+s}{'}\PYG{l+s}{link\PYGZus{}text}\PYG{l+s}{'}\PYG{p}{:} \PYG{l+s}{'}\PYG{l+s}{Add \PYGZob{}friendly\PYGZus{}name\PYGZcb{}}\PYG{l+s}{'}\PYG{p}{,} \PYG{l+s}{'}\PYG{l+s}{link\PYGZus{}url}\PYG{l+s}{'}\PYG{p}{:} \PYG{l+s}{'}\PYG{l+s}{add}\PYG{l+s}{'}\PYG{p}{\PYGZcb{}}\PYG{p}{,}
       \PYG{p}{]}
\end{Verbatim}


\item {} 
\textbf{list\_per\_record\_actions} -- 
list of actions that are to be displayed for each row. These can contain a special keyword PK for referring to the primary key value(s) for the current record. Example:

\begin{Verbatim}[commandchars=\\\{\}]
\PYG{n}{list\PYGZus{}per\PYGZus{}record\PYGZus{}actions} \PYG{o}{=} \PYG{p}{[}
        \PYG{p}{\PYGZob{}}\PYG{l+s}{'}\PYG{l+s}{link\PYGZus{}text}\PYG{l+s}{'}\PYG{p}{:} \PYG{l+s}{'}\PYG{l+s}{Details}\PYG{l+s}{'}\PYG{p}{,} \PYG{l+s}{'}\PYG{l+s}{link\PYGZus{}url}\PYG{l+s}{'}\PYG{p}{:} \PYG{l+s}{'}\PYG{l+s}{details/\PYGZob{}PK\PYGZcb{}}\PYG{l+s}{'}\PYG{p}{\PYGZcb{}}\PYG{p}{,}
        \PYG{p}{\PYGZob{}}\PYG{l+s}{'}\PYG{l+s}{link\PYGZus{}text}\PYG{l+s}{'}\PYG{p}{:} \PYG{l+s}{'}\PYG{l+s}{Edit}\PYG{l+s}{'}\PYG{p}{,} \PYG{l+s}{'}\PYG{l+s}{link\PYGZus{}url}\PYG{l+s}{'}\PYG{p}{:} \PYG{l+s}{'}\PYG{l+s}{edit/\PYGZob{}PK\PYGZcb{}}\PYG{l+s}{'}\PYG{p}{\PYGZcb{}}\PYG{p}{,}
        \PYG{p}{\PYGZob{}}\PYG{l+s}{'}\PYG{l+s}{link\PYGZus{}text}\PYG{l+s}{'}\PYG{p}{:} \PYG{l+s}{'}\PYG{l+s}{Delete}\PYG{l+s}{'}\PYG{p}{,} \PYG{l+s}{'}\PYG{l+s}{link\PYGZus{}url}\PYG{l+s}{'}\PYG{p}{:} \PYG{l+s}{'}\PYG{l+s}{delete/\PYGZob{}PK\PYGZcb{}}\PYG{l+s}{'}\PYG{p}{\PYGZcb{}}\PYG{p}{,}
       \PYG{p}{]}
\end{Verbatim}


\item {} 
\textbf{detail\_actions} -- 
list of actions to be displayed on the details page, similar to \textbf{list\_per\_record\_actions}. Example:

\begin{Verbatim}[commandchars=\\\{\}]
\PYG{n}{detail\PYGZus{}actions} \PYG{o}{=} \PYG{p}{[}
        \PYG{p}{\PYGZob{}}\PYG{l+s}{'}\PYG{l+s}{link\PYGZus{}text}\PYG{l+s}{'}\PYG{p}{:} \PYG{l+s}{'}\PYG{l+s}{Edit}\PYG{l+s}{'}\PYG{p}{,} \PYG{l+s}{'}\PYG{l+s}{link\PYGZus{}url}\PYG{l+s}{'}\PYG{p}{:} \PYG{l+s}{'}\PYG{l+s}{../edit/\PYGZob{}PK\PYGZcb{}}\PYG{l+s}{'}\PYG{p}{\PYGZcb{}}\PYG{p}{,}
        \PYG{p}{\PYGZob{}}\PYG{l+s}{'}\PYG{l+s}{link\PYGZus{}text}\PYG{l+s}{'}\PYG{p}{:} \PYG{l+s}{'}\PYG{l+s}{Delete}\PYG{l+s}{'}\PYG{p}{,} \PYG{l+s}{'}\PYG{l+s}{link\PYGZus{}url}\PYG{l+s}{'}\PYG{p}{:} \PYG{l+s}{'}\PYG{l+s}{../delete/\PYGZob{}PK\PYGZcb{}}\PYG{l+s}{'}\PYG{p}{\PYGZcb{}}\PYG{p}{,}
       \PYG{p}{]}
\end{Verbatim}


\end{itemize}

\end{description}\end{quote}

\textbf{TODO}
\begin{itemize}
\item {} 
More documentation of various options and methods

\item {} 
A CRUDController tutorial

\item {} 
Tests for the controller

\item {} 
Add support for composite primary keys

\item {} 
Once CRDUController is complete, may be put table display login in a ModelTable component??

\end{itemize}
\index{add() (pyck.controllers.CRUDController method)}

\begin{fulllineitems}
\phantomsection\label{pyck-controllers:pyck.controllers.CRUDController.add}\pysiglinewithargsret{\bfcode{add}}{}{}
The add record view

\end{fulllineitems}

\index{delete() (pyck.controllers.CRUDController method)}

\begin{fulllineitems}
\phantomsection\label{pyck-controllers:pyck.controllers.CRUDController.delete}\pysiglinewithargsret{\bfcode{delete}}{}{}
The record delete view

TODO:
\begin{itemize}
\item {} 
Later may need to add support for composite primary keys here.

\end{itemize}

\end{fulllineitems}

\index{details() (pyck.controllers.CRUDController method)}

\begin{fulllineitems}
\phantomsection\label{pyck-controllers:pyck.controllers.CRUDController.details}\pysiglinewithargsret{\bfcode{details}}{}{}
The record details view

\end{fulllineitems}

\index{edit() (pyck.controllers.CRUDController method)}

\begin{fulllineitems}
\phantomsection\label{pyck-controllers:pyck.controllers.CRUDController.edit}\pysiglinewithargsret{\bfcode{edit}}{}{}
The edit and update record view

\end{fulllineitems}

\index{list() (pyck.controllers.CRUDController method)}

\begin{fulllineitems}
\phantomsection\label{pyck-controllers:pyck.controllers.CRUDController.list}\pysiglinewithargsret{\bfcode{list}}{}{}
The listing view - Lists all the records with pagination

\end{fulllineitems}


\end{fulllineitems}

\index{add\_crud\_handler() (in module pyck.controllers)}

\begin{fulllineitems}
\phantomsection\label{pyck-controllers:pyck.controllers.add_crud_handler}\pysiglinewithargsret{\code{pyck.controllers.}\bfcode{add\_crud\_handler}}{\emph{config}, \emph{route\_name\_prefix='`}, \emph{url\_pattern\_prefix='`}, \emph{handler\_class=None}}{}
A utility function to quickly add all crud related routes and set them to the crud handler class with one function call, for example:

\begin{Verbatim}[commandchars=\\\{\}]
from pyck.controllers import add\_crud\_handler
from controllers.views import WikiCRUDController
.
.
.
add\_crud\_handler(config, APP\_NAME + '.', '/crud', WikiCRUDController)
\end{Verbatim}
\begin{quote}\begin{description}
\item[{Parameters}] \leavevmode\begin{itemize}
\item {} 
\textbf{config} -- The application config object

\item {} 
\textbf{route\_name\_prefix} -- Optional string prefix to add to all route names. Useful if you're adding multiple CRUD controllers and want to avoid route name conflicts.

\item {} 
\textbf{url\_pattern\_prefix} -- Optional string prefix to add to all crud related url patterns

\item {} 
\textbf{exclude} -- An optional iterable with the property names that should be excluded
from the form. All other properties will have fields.

\item {} 
\textbf{handler\_class} -- The handler class that is used to handle CRUD requests. Must be sub-class of {\hyperref[pyck-controllers:pyck.controllers.CRUDController]{\code{pyck.controllers.CRUDController}}}

\end{itemize}

\end{description}\end{quote}

\end{fulllineitems}



\chapter{Indices and tables}
\label{index:indices-and-tables}\begin{itemize}
\item {} 
\emph{genindex}

\item {} 
\emph{modindex}

\item {} 
\emph{search}

\end{itemize}


\renewcommand{\indexname}{Python Module Index}
\begin{theindex}
\def\bigletter#1{{\Large\sffamily#1}\nopagebreak\vspace{1mm}}
\bigletter{p}
\item {\texttt{pyck.controllers}}, \pageref{pyck-controllers:module-pyck.controllers}
\item {\texttt{pyck.forms}}, \pageref{pyck-forms:module-pyck.forms}
\end{theindex}

\renewcommand{\indexname}{Index}
\printindex
\end{document}
